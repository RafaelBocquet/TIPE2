\documentclass[12pt, a4paper]{article}

\usepackage[utf8]{inputenc}
\usepackage[french]{babel}
\usepackage[T1]{fontenc}
\usepackage[a4paper, margin=0.6in]{geometry}
\usepackage{listings}
\usepackage{color}
\usepackage{tcolorbox}
\usepackage{mathtools}
\usepackage{comment}
\usepackage{psfrag}
\usepackage{graphviz}
\usepackage{etoolbox}
\usepackage[nomessages]{fp}

% Title

\title{Base de preuves pour les types dépendents}
\author{Rafaël Bocquet}
\date{}

% Macros to size the graph nodes depending on the content.

\newcommand*{\getlength}[2]{
  \FPeval\length{1.1 * \number#2 / 4718592.0}
  \csedef{#1}{\length}
}

\newlength{\getwidthlengthw}
\newlength{\getwidthlengthh}

\newcommand*{\getwidthlength}[2]{
  \settowidth{\getwidthlengthw}{#2}
  \getlength{#1width}{\getwidthlengthw}
  \settoheight{\getwidthlengthh}{#2}
  \getlength{#1height}{\getwidthlengthh}
}

\newcommand*{\nodepsfrag}[2]{
  \psfrag{#1}[cc][cc]{#2}
  \getwidthlength{#1}{#2}
}

% Begin document

\begin{document}

\maketitle
\tableofcontents

\newpage

\section{Position du problème}
\subsection{$\lambda \equiv P$}
\subsection{Recherche de preuves triviales}
\section{Implementations possibles}
\subsection{Prolog-like auto tactic} $\rightarrow$ Can't scale
\subsection{Recherche exacte / isomorphismes} $\rightarrow$ Less powerfull, more scalable
\section{Implementation}
\subsection{Normalisation d'un type + gamma}
\subsection{Normalisation d'un type + gamma / isomorphismes}
\subsection{Recherche du type normalisé dans la base}
\subsection{Construction de la preuve dans l'environment donné}
\section{Further}
\subsection{Integration avec un systeme de types réel}

% 
identityCongTy : \\
\begin{math}
(A : \top) \rightarrow (a \, b : A) \rightarrow (a = b) \rightarrow (B : \top) \rightarrow (f : A \rightarrow B) \rightarrow (f a = f b)
\end{math} \\ \\
\begin{math}
\Pi{(a : \top)} . \Pi{(b : a)} . \Pi{(c : a)} . \\ \Pi{(d : \Pi{(d : \Pi{(d : a)} . \top)} . \Pi{(e : (d b))} . (d c))} . \\ \Pi{(e : \top)} . \Pi{(f : \Pi{(f : a)} . e)} . \Pi{(g : \Pi{(g : e)} . \top)} . \Pi{(h : (g (f b)))} . \\ (g (f c))
\end{math} \\
 normalises to \\

\begin{psfrags}
\nodepsfrag{n0}{$(E_{1} (E_{2} E_{5}))$}
\nodepsfrag{n1}{$(E_{1} (E_{2} E_{6}))$}
\nodepsfrag{n2}{$\Pi{(a : E_{3})} \rightarrow \top$}
\nodepsfrag{n3}{$\Pi{(a : E_{7})} \rightarrow E_{3}$}
\nodepsfrag{n4}{$\top$}
\nodepsfrag{n5}{$\Pi{(a : \Pi{(a : E_{7})} \rightarrow \top)} \rightarrow \Pi{(b : (a E_{6}))} \rightarrow (a E_{5})$}
\nodepsfrag{n6}{$E_{7}$}
\nodepsfrag{n7}{$E_{7}$}
\nodepsfrag{n8}{$\top$}
\expandafter\digraph G {
  graph[size=8];
  node[shape=rectangle];
  n0[width=\csuse{n0width}, height=\csuse{n0height}, color=red];
  n1[width=\csuse{n1width}, height=\csuse{n1height}];
  n2[width=\csuse{n2width}, height=\csuse{n2height}];
  n3[width=\csuse{n3width}, height=\csuse{n3height}];
  n4[width=\csuse{n4width}, height=\csuse{n4height}];
  n5[width=\csuse{n5width}, height=\csuse{n5height}];
  n6[width=\csuse{n6width}, height=\csuse{n6height}];
  n7[width=\csuse{n7width}, height=\csuse{n7height}];
  n8[width=\csuse{n8width}, height=\csuse{n8height}];
  n2 -> n0;
  n3 -> n0;
  n6 -> n0;
  n2 -> n1;
  n3 -> n1;
  n7 -> n1;
  n4 -> n2;
  n4 -> n3;
  n8 -> n3;
  n6 -> n5;
  n7 -> n5;
  n8 -> n5;
  n8 -> n6;
  n8 -> n7;
}
\end{psfrags}

\end{document}